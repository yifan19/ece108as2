\documentclass[a4paper,12pt]{article}
\usepackage{amsmath}
\usepackage[linguistics]{forest}
\usepackage{natded}
\usepackage{prooftrees}
\usepackage{tikz}
\usepackage{karnaugh-map}

\newcommand{\CommaPunct}{\mathpunct{\raisebox{0.5ex}{,}}}
\begin{document}

\title{ECE108 Assignment2} \author{Yi Fan Yu yf3yu@edu.uwaterloo.ca} \date{\today}
\maketitle

\pagenumbering{arabic}
\tableofcontents
\newpage
\renewcommand{\thesection}{\Roman{section}}
\section{Theorem 1}
 need to prove a theorem really quicky before I prove this:\\
\[\frac{\lnot (P_{1} \vee P_{2})}{\lnot P_{1}}\]
proof by models through semantic tableau:\\
\begin{forest}
[$\lnot (P_{1} \vee P_{2}):T \CommaPunct \lnot P_{1} : F$
  [$P_{1} : T$
    [$(P_{1} \vee P_{2}):F$
      [$(P_{1}:F) \CommaPunct (P_{2} : F)$
        [X]
      ]
    ]
  ]
]
\end{forest}
\pagebreak
\section{Theorem2}
\[ \frac{P \rightarrow Q}{ \lnot P \vee Q}
\]
\[
\KMproof{
  \cbblk{
  \proofline{(P \rightarrow Q) \rightarrow (\lnot P \vee Q) }{}
  }{
    \proofline{P \rightarrow Q}{1,CD}
     \cbblk{
     \proofline{(\lnot P \vee Q)}{subDer}
     }{
       \proofline{\lnot (\lnot P \vee Q)}{ID}
       \cbblk{
       \proofline{\lnot P}{subDerv}
       }{
          \proofline{P}{ID}
          \proofline{Q}{6,2,MP}
          \proofline{\lnot P \vee Q}{7,ADD}
          \proofline{\lnot(\lnot P \vee Q)}{4}
        } 
        \proofline{\lnot P \vee Q}{4,ADD}
        \proofline{\lnot (\lnot P \vee Q)}{4}
     }
  } 
}
\]
\pagebreak\\
\setcounter{section}{0}
\renewcommand{\thesection}{\arabic{section}}
\section{Formalizing an Argument}
\subsection{Big Bang Theory}
If the Big Bang Theory is correct then either there was a time before anything existed or the world will come to an end.  The world will not come to an end.  Therefore, if there was no time before anything existed, the Big Bang Theory is incorrect.
\[
P: \text{the Big Bang Theory is correct} 
\]
\[
Q: \text{there was a time before anything existed}
\]
\[
R: \text{the world wille come to an end} 
\]
\begin{equation} 
P \implies (Q \vee R) 
\end{equation}
\begin{equation} 
\neg R 
\end{equation}
\begin{equation} 
 \textbf{as a conclusion : } \neg Q \implies \neg P 
\end{equation}

\[
\KMproof{
  \cbblk{
  \proofline {\lnot Q \rightarrow \lnot P} {}
  }{
    \proofline{\lnot Q}{1,CD}
    \cbblk{
    \proofline{\lnot P}{}}{
      \proofline{P}{3,ID}
      \proofline{P \rightarrow (Q \vee R)}{P2}
      \proofline{Q \vee R}{4,5, MP}
      \proofline{\lnot R} {P1} 
      \proofline{Q}{5,6,MTP}
      \proofline{\lnot Q}{2}
    }
  }
}
\]

\pagebreak
\subsection{To Win a Gold Medal}
To win a gold medal, an athlete must be very fit.  If s/he does not win a gold medal, then either s/he arrived late for the competition or his/her training was interrupted.  If s/he is not very fit, s/he will blame his/her coach.  If s/he blames his/her coach, or his/her training is interrupted, then s/he will will not get into the competition.  Therefore, if s/he gets into the competition, s/he will not have arrived late.
\[
P: \text{win a gold medal} 
\]
\[
Q: \text{an athlete must be very fit} 
\]
\[
R: \text{s/he arrived late for the competition } 
\]
\[
S: \text{his/her training was interrupted.} 
\]
\[
T: \text{s/he will blame his/her coach}
\]
\[
U: \text{if s/he gets into the competition}
\]
\begin{equation} 
P \implies Q 
\end{equation}
\begin{equation} 
\neg P \implies (R \vee S)
\end{equation}
\begin{equation} 
\neg Q \implies T
\end{equation}
\begin{equation} 
(T \vee S ) \implies \neg U
\end{equation}
\begin{equation} 
\textbf{as conclusion: } U \implies \neg R
\end{equation}\\
Logic is not sound\\
I can find a falsifying assignment:\\
$P:True \CommaPunct Q:True \CommaPunct R:True \CommaPunct T:False \CommaPunct S:False \CommaPunct  U:True$\\
The conclusion is false since:\\
$U \rightarrow \lnot R : F$\\
because $True \rightarrow False : False$\\
But all the premises are true:\\
$P$ is false so $P \rightarrow Q$ is True\\
$R$ is true so $ \neg P \rightarrow(R \vee S)$ is True\\
$Q$ us true so $ \neg Q \rightarrow T $ is also True\\
$T$ and $S$ are both False so $ (T \vee S) \rightarrow \lnot U$ is True\\

\pagebreak

\subsection{Hector and the Battle of Priam} 
If Hector wins the battle, he will plunder the city.  If he does not win the battle, he will either be killed or go into exile.  If he plunders the city, then Priam will lose his kingdom.  If Priam loses his kingdom or Hector goes into exile, then the war will end.  Therefore, if the war does not end, Hector will be killed. 
\[
P: \text{Hector wins the battle} 
\]
\[
Q: \text{Hector plunders the city} 
\]
\[
R: \text{Hector is killed} 
\]
\[
S: \text{Hector goes into exile}
\]
\[
T: \text{Priam will lose his kingdom}
\]
\[
U: \text{war will end}
\]

\begin{equation} 
P \implies Q 
\end{equation}
\begin{equation} 
\neg P \implies (R \vee S)
\end{equation}
\begin{equation} 
Q \implies T
\end{equation}
\begin{equation} 
(T \vee S) \implies U
\end{equation}
\begin{equation} 
\textbf{as conclusion: } \neg U \implies R
\end{equation}\\

\[
\KMproof{
  \cbblk{
  \proofline{\lnot U \rightarrow R}{}
  }{
    \proofline{\lnot U}{1,CD}
    \proofline{(T \vee S) \rightarrow U }{P12}
    \proofline{\lnot (T \vee S)}{2,3,MT}
    \proofline{\lnot T}{4,Theorem1}
    \proofline{\lnot S}{4,Theorem1}
    \proofline{Q \rightarrow T}{P11}
    \proofline{\lnot Q}{5,7,MT}
    \proofline{P \rightarrow Q}{P9}
    \proofline{\lnot P}{8,9,MT}
    \proofline{\lnot P \rightarrow (R \vee S)}{P10}
    \proofline{(R \vee S)}{10,11,MP}
    \proofline{R}{6,12,MTP}
    } 
  }
\]
\pagebreak
\subsection{colonel and the murder}
If the colonel was out of the room when the murder was committed then he couldn't have been right about the weapon used.  Either the butler is lying or he knows who the murderer was.  If Lady Barntree was not the murderer then either the colonel was in the room at the time or or the butler is lying.  Either the butler knows who the murderer was or the colonel was out of the room at the time of the murder.  Therefore, if the colonel was right about the weapon then Lady Barntree was the murderer.

\[
P: \text{the colonel was out of the room when the murder was committed}  
\]
\[
Q: \text{colonel couldn't have been right about the weapon used} 
\]
\[
R: \text{the butler is lying} 
\]
\[
S: \text{the butler knows who the murderer was}
\]
\[
T: \text{Lady Barntree was not the murderer} 
\]

\begin{equation} 
P \implies Q 
\end{equation}
\begin{equation} 
\neg(R \iff S)
\end{equation}
\begin{equation} 
T \implies (\neg P \vee R)
\end{equation}
\begin{equation} 
\neg(S \iff P)
\end{equation}
\begin{equation} 
\textbf{as conclusion: } \neg Q \implies \neg T
\end{equation}\\
Logic is not sound\\
I can find a falsifying assignment:\\
$P:False \CommaPunct Q:False \CommaPunct R:False\CommaPunct T:True \CommaPunct S:True$\\
The conclusion is false since:\\
$\lnot Q \rightarrow \lnot T : F$\\
because $True \rightarrow False : False$\\
But all the premises are true:\\
$P$ is False so $P \rightarrow Q$ is True\\
$R$ is False and $S$ is True so $\neg(S \iff P)$ is True\\
$T$ and $\lnot P$ are both True so $T \implies (\neg P \vee R)$ is True\\
$S$ is True and $P$ is False $\neg(S \iff P)$ is True\\
\pagebreak
\section{Tautologies and Friends}
\subsection{Q1}
$(P \wedge Q) \rightarrow (P \rightarrow Q)$\\

\begin{displaymath}
\begin {array} {|c c |c|c|c|}
P & Q & P \wedge Q &P \rightarrow Q & (P \wedge Q) \rightarrow (P \rightarrow Q)\\
\hline
F & F & F & T & T \\
F & T & F & T & T \\
T & F & F & F & T \\
T & T & T & T & T \\
\end{array}
\end{displaymath}
Tautology\\
\subsection{Q2}
$(P \wedge Q) \leftrightarrow (P \rightarrow Q)$\\
\begin{displaymath}
\begin {array} {|c c |c|c|c|}
P & Q & P \wedge Q &P \rightarrow Q & (P \wedge Q) \iff (P \rightarrow Q)\\
\hline
F & F & F & T & F \\
F & T & F & T & F \\
T & F & F & F & T \\
T & T & T & T & T \\

\end{array}
\end{displaymath}
not Tautology\\

\subsection{Q3}
$(\lnot P \vee Q) \rightarrow (P \rightarrow \lnot Q)$\\
\begin{displaymath}
\begin {array} {|c c |c|c|c|}
P & Q & \lnot P \vee Q &P \rightarrow \lnot Q
&(\lnot P \vee Q) \rightarrow (P \rightarrow \lnot Q)\\
\hline
F & F & T & T & T \\
F & T & T & T & T \\
T & F & F & T & T \\
T & T & T & F & F \\ 
\end{array}
\end{displaymath} 
not Tautology\\
\pagebreak

\subsection{Q4}
$(((P \rightarrow Q) \rightarrow P) \rightarrow Q)$\\

\begin{displaymath}
\begin {array} {|c c |c|c|c|}
P & Q & (P \rightarrow Q) & ((P \rightarrow Q) \rightarrow P)
&(((P \rightarrow Q) \rightarrow P) \rightarrow Q)\\
\hline
F & F & T & F & T \\
F & T & T & F & T \\
T & F & F & T & F \\
T & T & T & T & T \\

\end{array}
\end{displaymath}
not Tautology
\subsection{Q5}
$(P \rightarrow (Q \rightarrow (P \rightarrow Q)))$\\
\begin{displaymath}
\begin {array} {|c c |c|c|c|}
P & Q & (P \rightarrow Q) & (Q \rightarrow (P \rightarrow Q) 
&(P \rightarrow (Q \rightarrow (P \rightarrow Q)))\\
\hline
F & F & T & T & T \\
F & T & T & T & T \\
T & F & F & T & T \\
T & T & T & T & T \\

\end{array}
\end{displaymath}
Tautology\\

\subsection{Q6}
$((P \wedge \lnot Q) \rightarrow \lnot R) \leftrightarrow ((P \wedge R) \rightarrow Q)$\\
\begin{displaymath}
\begin {array} {|c c c|c|c|c|c|c|c|}
P & Q & R & (P \wedge \lnot Q) & 
((P \wedge \lnot Q) \rightarrow \lnot R)&
P \wedge R) &
(P \wedge R) \rightarrow Q\\
\hline
F & F & F & F & T & F & T\\
F & F & T & F & T & F & T\\
F & T & F & F & T & F & T\\
F & T & T & F & T & F & T\\
T & F & F & T & T & F & T\\
T & F & T & T & F & T & F\\
T & T & F & F & T & F & T\\
T & T & T & F & T & T & T\\
\end{array}
\end{displaymath}
we see that column 3 = column 5\\
the if and only if is true, so tautology\\
\pagebreak
\subsection{Q7}
$(((P \vee Q) \vee R) \vee S) \leftrightarrow (P \vee (Q \vee (R \vee S)))$\\
\begin{displaymath}
\begin {array} {|c c c c|c|c|c|}
P & Q & R & S &
P \vee Q &
P \vee Q) \vee R &
((P \vee Q) \vee R )\vee S \\
\hline
F & F & F & F & F & F & F\\
F & F & F & T & F & F & T\\
F & F & T & F & F & T & T\\
F & F & T & T & F & T & T\\
F & T & F & F & T & T & T\\
F & T & F & T & T & T & T\\
F & T & T & F & T & T & T\\
F & T & T & T & T & T & T\\
T & F & F & F & T & T & T\\
T & F & F & T & T & T & T\\
T & F & T & F & T & T & T\\
T & F & T & T & T & T & T\\
T & T & F & F & T & T & T\\
T & T & F & T & T & T & T\\
T & T & T & F & T & T & T\\
T & T & T & T & T & T & T\\
\end{array}
\end{displaymath}

\begin{displaymath}
\begin {array} {|c|c|c|}
R \vee S &
Q \vee (R \vee S) &
P \vee (Q \vee (R \vee S))\\
\hline

F & F & F\\
T & T & T\\
T & T & T\\
T & T & T\\
F & T & T\\
T & T & T\\
T & T & T\\
T & T & T\\
F & F & T\\
T & T & T\\
T & T & T\\
T & T & T\\
F & T & T\\
T & T & T\\
T & T & T\\
T & T & T\\
\end{array}
\end{displaymath}

as anyone can clearly see from commutativity of OR, this is a tautology
\pagebreak
\subsection{Q8}
$(((P \rightarrow Q) \rightarrow R) \rightarrow S) \leftrightarrow (P \rightarrow (Q \rightarrow (R \rightarrow S)))$\\
\begin{displaymath}
\begin {array} {|c c c c|c|c|c|c|c}
P & Q & R & S & 
P \rightarrow Q &
(P \rightarrow Q) \rightarrow R &
((P \rightarrow Q) \rightarrow R) \rightarrow S \\
\hline

F & F & F & F & T & F & T\\
F & F & F & T & T & F & T\\
F & F & T & F & T & T & F\\
F & F & T & T & T & T & T\\
F & T & F & F & T & F & T\\
F & T & F & T & T & F & T\\
F & T & T & F & T & T & F\\
F & T & T & T & T & T & T\\
T & F & F & F & F & T & F\\
T & F & F & T & F & T & T\\
T & F & T & F & F & T & F\\
T & F & T & T & F & T & T\\
T & T & F & F & T & F & T\\
T & T & F & T & T & F & T\\
T & T & T & F & T & T & F\\
T & T & T & T & T & T & T\\

\end{array}
\end{displaymath}

\begin{displaymath}
\begin {array} {|c c c c|c|c|c|c|c}
P & Q & R & S & 
R \rightarrow S &
Q \rightarrow(R \rightarrow S) &
P \rightarrow (Q \rightarrow(R \rightarrow S))\\
\hline

F & F & F & F & T & T & T\\
F & F & F & T & T & T & T\\
F & F & T & F & F & T & T\\
F & F & T & T & T & T & T\\
F & T & F & F & T & T & T\\
F & T & F & T & T & T & T\\
F & T & T & F & F & F & T\\
F & T & T & T & T & T & T\\
T & F & F & F & T & T & T\\
T & F & F & T & T & T & T\\
T & F & T & F & F & T & T\\
T & F & T & T & T & T & T\\
T & T & F & F & T & T & T\\
T & T & F & T & T & T & T\\
T & T & T & F & F & F & F\\
T & T & T & T & T & T & T\\

\end{array}
\end{displaymath}
after brainlessly bruteforce everything in a truth table, we see that this is not a tautology.\\
\subsection{Q9}
$(P \rightarrow (\lnot R \rightarrow \lnot S)) \vee ((S \rightarrow (P \vee \lnot T)) \vee (\lnot Q \rightarrow R))$\\

\begin{displaymath}
\begin {array} {|c c c c c|c|c|c|c|c|}
P & Q & R & S & T &
\lnot R \rightarrow \lnot S & 
(P \rightarrow (\lnot R \rightarrow \lnot S)) \\
\hline

F & F & F & F & F & T & T\\
F & F & F & F & T & T & T\\
F & F & F & T & F & F & T\\
F & F & F & T & T & F & T\\
F & F & T & F & F & T & T\\
F & F & T & F & T & T & T\\
F & F & T & T & F & T & T\\
F & F & T & T & T & T & T\\
F & T & F & F & F & T & T\\
F & T & F & F & T & T & T\\
F & T & F & T & F & F & T\\
F & T & F & T & T & F & T\\
F & T & T & F & F & T & T\\
F & T & T & F & T & T & T\\
F & T & T & T & F & T & T\\
F & T & T & T & T & T & T\\
T & F & F & F & F & T & T\\
T & F & F & F & T & T & T\\
T & F & F & T & F & F & F\\
T & F & F & T & T & F & F\\
T & F & T & F & F & T & T\\
T & F & T & F & T & T & T\\
T & F & T & T & F & T & T\\
T & F & T & T & T & T & T\\
T & T & F & F & F & T & T\\
T & T & F & F & T & T & T\\
T & T & F & T & F & F & F\\
T & T & F & T & T & F & F\\
T & T & T & F & F & T & T\\
T & T & T & F & T & T & T\\
T & T & T & T & F & T & T\\
T & T & T & T & T & T & T\\
\end{array}
\end{displaymath}
ok now, since the rest is linked by OR $\vee$, we only need
to find 4 cases where the first statement doesn't evaluate to true.

\begin{displaymath}
\begin {array} {|c c c c c|c|c|c|c|c|}
P & Q & R & S & T &
(P \rightarrow (\lnot R \rightarrow \lnot S)) &
P \vee T \\
\hline

T & F & F & T & F & F& T\\
T & F & F & T & T & F& T\\
T & T & F & T & F & F& T\\
T & T & F & T & T & F& T\\

\end{array}
\end{displaymath}

well we see in $S \rightarrow (P \vee T)$  that if P is True, then the whole thing is true no matter T or S\\

therefore this is a tautology
\pagebreak
\section{Semantic Tableaux}
\subsection{Q1}
$((P \wedge Q) \rightarrow (P \rightarrow Q))$\\
\begin{forest}
[
$((P \wedge Q) \rightarrow (P \rightarrow Q)): F$
[$(P \wedge Q): T \CommaPunct (P \rightarrow Q): F$
[$ P :T \CommaPunct Q: F$
[$ P :T \CommaPunct Q: T$
[X]
]
]
]
]
\end{forest}\\
Tautology\\

\subsection{Q2}
$(P \wedge Q) \leftrightarrow (P \rightarrow Q)$\\

\begin{forest}
[ $(P \wedge Q) \leftrightarrow (P \rightarrow Q): F$
[$(P \wedge Q): T \CommaPunct (P \rightarrow Q): F$
  [$ P :T \CommaPunct Q: F$
    [$ P :T \CommaPunct Q: T$
      [X]
    ]
  ]
]
[$(P \wedge Q): F \CommaPunct (P \rightarrow Q): T$
  [$P : F$  
    [$P:F \CommaPunct Q:T$
      [O]
    ]
  ]
  [$Q : F$]  
]
]
\end{forest}\\
Neither\\
\subsection{Q3}
$(\lnot P \vee Q) \rightarrow (P \rightarrow \lnot Q)$\\
\begin{forest}
[ $(\lnot P \vee Q) \rightarrow (P \rightarrow \lnot Q): F$
  [$(\lnot P \vee Q):T \CommaPunct (P \rightarrow \lnot Q): F$
    [$ P :T \CommaPunct \lnot Q: F$
      [$Q:T$
        [$\lnot P :T$
          [$ P :F $
            [X]
          ]
        ]
        [$ Q : T$
          [O]
        ]
      ]
    ]
  ]
]
\end{forest}\\
Neither\\
\subsection{Q4}
$(((P \rightarrow Q) \rightarrow P) \rightarrow Q)$\\

\begin{forest}
[$(((P \rightarrow Q) \rightarrow P) \rightarrow Q): F$
  [$(((P \rightarrow Q) \rightarrow P):T \CommaPunct Q: F$
    [$ P \rightarrow Q :F $
      [$P:T \CommaPunct Q:F$
        [O]
      ]
    ]
    [$ P :T$
      [O]
    ]
  ]
]
\end{forest}\\

actually not a contradiction because of the truth table.
Neither actually\\
\subsection{Q5}
$(P \rightarrow (Q \rightarrow (P \rightarrow Q)))$\\
\begin{forest}
[$(P \rightarrow (Q \rightarrow (P \rightarrow Q))): F$
  [$P:T \CommaPunct (Q \rightarrow (P \rightarrow Q)): F$
    [$Q:T \CommaPunct P\rightarrow Q :F$
      [$P:T \CommaPunct Q:F$
        [X]
      ]
    ]
  ]
]
\end{forest}\\
Tautology\\
\pagebreak
\subsection{Q6}
$((P \wedge \lnot Q) \rightarrow \lnot R) \leftrightarrow ((P \wedge R) \rightarrow Q)$\\

\begin{forest}
[$((P \wedge \lnot Q) \rightarrow \lnot R) \leftrightarrow ((P \wedge R) \rightarrow Q))): F$
  [$((P \wedge \lnot Q) \rightarrow \lnot R):T \CommaPunct ((P \wedge R) \rightarrow Q))): F$
    [$ P\wedge R : T \CommaPunct Q: F$
      [$ P :T \CommaPunct R:T$
        [$P \wedge \lnot Q : F$
          [$P:F$
            [X]
          ]
          [$\lnot Q:F$
            [$Q:T$
              [X]
            ]
          ]
        ]
        [$\lnot R : T$
          [$R:F$
            [X]
          ]
        ]
      ]
    ]
  ]
  [$((P \wedge \lnot Q) \rightarrow \lnot R):F \CommaPunct ((P \wedge R) \rightarrow Q))): T$
    [$ (P \wedge \lnot Q) :T \CommaPunct \lnot R : F$
      [ $R:T$
        [$P:T \CommaPunct \lnot Q:T$
          [$Q:F$
            [$(P \wedge R ):F$
              [$P:F$
                [X]
              ]
              [$R:F$
                [X]
              ]
            ]
            [$Q :T$
              [X] 
            ]
          ]
        ]
      ]
    ]
  ]
]
\end{forest}\\
Tautology\\
\subsection{Q7}
$(((P \vee Q) \vee R) \vee S) \leftrightarrow (P \vee (Q \vee (R \vee S)))$\\
\begin{forest}
[$(((P \vee Q) \vee R) \vee S) \leftrightarrow (P \vee (Q \vee (R \vee S))):F$
  [$(((P \vee Q) \vee R) \vee S):T \CommaPunct (P \vee (Q \vee (R \vee S))): F$
    [$P:F \CommaPunct (Q \vee (R \vee S)):F$
      [$Q:F \CommaPunct (R \vee S):F$
        [$R:F \CommaPunct S:F$
          [$S:T$
            [X]
          ]
          [$((P \vee Q) \vee R):T$
            [$R:T$
              [X]
            ]
            [$(P\vee Q)$
              [$P:T$
                [X]
              ]
              [$Q:T$
                [X]
              ]
            ]
          ]
        ] 
      ]
    ]
  ]
  [$(((P \vee Q) \vee R) \vee S):F$...
    [...$(P \vee (Q \vee (R \vee S))): T$
      [$S:F \CommaPunct ((P \vee Q) \vee R) :F$
        [$R:F \CommaPunct (P \vee Q):F$
          [$P:F \CommaPunct Q:F$
            [$R:T$
              [X]
            ]
            [$(Q \vee (R \vee S)):T$
              [$Q:T$
                [X]
              ]
              [$(R\vee S)$
                [$R:T$
                  [X]
                ]
                [$S:T$
                  [X]
                ]
              ]
            ]
          ] 
        ]
      ]
    ]
  ]
]
\end{forest}\\
Tautology\\

\pagebreak
\subsection{Q8}
$(((P \rightarrow Q) \rightarrow R) \rightarrow S) \leftrightarrow (P \rightarrow (Q \rightarrow (R \rightarrow S)))$\\

\begin{forest}
[$(((P \rightarrow Q) \rightarrow R) \rightarrow S) \leftrightarrow (P \rightarrow (Q \rightarrow (R \rightarrow S))):F$
  [$(((P \rightarrow Q) \rightarrow R) \rightarrow S):T \CommaPunct$
    [...$(P \rightarrow (Q \rightarrow (R \rightarrow S))): F$
      [$P :T \CommaPunct (Q \rightarrow (R \rightarrow S)):F$
        [similarly $Q:T \CommaPunct R:T \CommaPunct S:F$
          [$(((P \rightarrow Q) \rightarrow R) \rightarrow S):F$ 
            [X if you plug in values]
          ]
        ]
      ]
    ]
  ]
  [$(((P \rightarrow Q) \rightarrow R) \rightarrow S):F$
    [...$(P \rightarrow (Q \rightarrow (R \rightarrow S))): T$
      [$((P \rightarrow Q) \rightarrow R):T \CommaPunct S:F$
        [$(P \rightarrow Q) :F$
          [$P:T \CommaPunct Q:F$
            [$P:F$
              [X]
            ]
            [$(Q \rightarrow (R \rightarrow S)):T$
              [$Q:F$
                [O]
              ] 
              [$R \rightarrow S :T$]
            ]
          ]
        ]
        [$R:T$]
      ]
    ]
  ]
]
\end{forest}\\
Neither\\
\pagebreak
\subsection{Q9}
$(P \rightarrow (\lnot R \rightarrow \lnot S)) \vee ((S \rightarrow (P \vee \lnot T)) \vee (\lnot Q \rightarrow R))$ \\

\begin{forest}
[$(P \rightarrow (\lnot R \rightarrow \lnot S)) \vee ((S \rightarrow (P \vee \lnot T)) \vee (\lnot Q \rightarrow R)) : F$
  [$(P \rightarrow (\lnot R \rightarrow \lnot S)):F$ ...
    [...$((S \rightarrow (P \vee \lnot T)) \vee (\lnot Q \rightarrow R)):F$
      [$(S \rightarrow (P \vee \lnot T)):F \CommaPunct (\lnot Q \rightarrow R):F$
        [$ \lnot Q :True \CommaPunct R: F$
          [$ Q:F$
            [$ S:True \CommaPunct (P \vee \lnot T) : F$
              [$ P:True \CommaPunct (\lnot R \rightarrow \lnot S) : F$
                [ $\lnot R : True \CommaPunct \lnot S: F$
                  [ $R:F \CommaPunct S:True$
                    [$P:F \CommaPunct \lnot T : True$
                      [X]
                    ]
                  ]
                ]
              ]
            ]
          ]
        ]
      ]
    ]
  ]
]
\end{forest}\\
Tautology\\
\pagebreak

\section{Kalish-Montegue Derivations}
\subsection{Q1}
$(P \wedge Q) \rightarrow (P \rightarrow Q)$\\

\[
\KMproof{
  \cbblk{
  \proofline{(P \wedge Q) \rightarrow (P \rightarrow Q)}{}
  }{
    \proofline{P \wedge Q}{1,CD}
    \cbblk{
     \proofline{P \rightarrow Q}{}
   }{
     \proofline{P}{3,CD}
     \proofline{P}{2,SIMP}
     \proofline{Q}{2,SIMP}
    }
 } 
}
\]
\subsection{Q5}
$(P \rightarrow (Q \rightarrow (P \rightarrow Q)))$\\

\[
\KMproof{
  \cbblk{
  \proofline{(P \rightarrow (Q \rightarrow (P \rightarrow Q)))}{}
  }{
    \proofline{P}{1,CD}
     \cbblk{
     \proofline{Q \rightarrow (P \rightarrow Q) }{1,subDer}
   }{
      \proofline{Q}{3,CD}
      \cbblk{
        \proofline{P \rightarrow Q} {3,subDerv}
       }{
        \proofline{P}{5,CD}
        \proofline{Q}{4}
        }
    }
  } 
}
\]

$((P \wedge \lnot Q) \rightarrow \lnot R) \leftrightarrow ((P \wedge R) \rightarrow Q)$\\
\subsection{Q6}
\subsubsection{Forward}
\[
\KMproof{
  \cbblk{
  \proofline{((P \wedge \lnot Q) \rightarrow \lnot R) \rightarrow ((P \wedge R) \rightarrow Q)}{}
  }{
    \proofline{(P \wedge \lnot Q) \rightarrow \lnot R}{1,CD}
     \cbblk{
     \proofline{(P \wedge R) \rightarrow Q }{2,subDer}
   }{
      \proofline{P \wedge R}{3,CD}
      \proofline{P}{4,SIMP}
      \proofline{R}{4,SIMP}
      \proofline{\lnot \lnot R}{6,DN}
      \proofline{\lnot (P \wedge \lnot Q)} {2,7,MT}
      \cbblk{
        \proofline{Q} {8,subDerv}
       }{
        \proofline{\lnot Q}{9,ID}
        \proofline{P \wedge \lnot Q}{5,10,ADJ}
        \proofline{\lnot (P \wedge \lnot Q)}{8}
        }
    }
  } 
}
\]
\subsubsection{Backward}
\[
\KMproof{
  \cbblk{
  \proofline{((P \wedge \lnot Q) \rightarrow \lnot R) \leftarrow ((P \wedge R) \rightarrow Q)}{}
  }{
    \proofline{(P \wedge R) \rightarrow Q }{1,CD}
     \cbblk{
     \proofline{(P \wedge \lnot Q) \rightarrow \lnot R}{subDer}
   }{
      \proofline{P \wedge \lnot Q}{3,CD}
      \proofline{P}{4,SIMP}
      \proofline{\lnot Q}{4,SIMP}
      \proofline{\lnot (P \wedge R)}{2,6,MT}
      \cbblk{
        \proofline{\lnot R} {subDerv}
       }{
        \proofline{R}{8,ID}
        \proofline{P \wedge R}{5,9,ADJ}
        \proofline{\lnot (P \wedge R)}{7}
        }
    }
  } 
}
\]
\pagebreak 
\subsection{Q7}
$(((P \vee Q) \vee R) \vee S) \leftrightarrow (P \vee (Q \vee (R \vee S)))$
\subsubsection{Theorem 1}
\[\frac{\lnot (P_{1} \vee P_{2})}{\lnot P_{1}}\]


\subsubsection{forward}
\[
\KMproof{
  \cbblk{
  \proofline{(((P \vee Q) \vee R) \vee S) \rightarrow (P \vee (Q \vee (R \vee S)))}{}
  }{
    \proofline{(((P \vee Q) \vee R) \vee S)}{1,CD}
     \cbblk{
     \proofline{(P \vee (Q \vee (R \vee S)))}{subDer}
   }{
      \proofline{\lnot (P \vee (Q \vee (R \vee S)))}{3,ID}
      \proofline{\lnot P}{4,Theorem 1}
      \proofline{\lnot (Q \vee (R \vee S))}{4,Theorem1}
      \proofline{\lnot Q}{6,Theorem1}
      \proofline{\lnot(R \vee S)}{6,Theorem1}
      \proofline{\lnot R }{8,Theorem1}
      \proofline{\lnot S }{8,Theorem1}
      \proofline{((P \vee Q) \vee R)}{2,10,MTP}
      \proofline{(P \vee Q) }{9,11,MTP}
      \proofline{P}{7,12,MTP}
    }
  } 
}
\]
\subsubsection{backward}
\[
\KMproof{
  \cbblk{
  \proofline{(((P \vee Q) \vee R) \vee S) \leftarrow (P \vee (Q \vee (R \vee S)))}{}
  }{
    \proofline{(P \vee (Q \vee (R \vee S)))}{1,CD}
     \cbblk{
     \proofline{(((P \vee Q) \vee R) \vee S)}{subDer}
   }{
      \proofline{\lnot (((P \vee Q) \vee R) \vee S)}{3,ID}
      \proofline{\lnot S}{4,Theorem 1}
      \proofline{\lnot ((P \vee Q) \vee R)}{4,Theorem1}
      \proofline{\lnot R}{6,Theorem1}
      \proofline{\lnot(P \vee Q)}{6,Theorem1}
      \proofline{\lnot P }{8,Theorem1}
      \proofline{\lnot Q }{8,Theorem1}
      \proofline{Q \vee (R \vee S)}{9,10,MTP}
      \proofline{(R \vee S) }{10,11,MTP}
      \proofline{R}{5,12,MTP}
    }
  } 
}
\]
\subsection{Q9}

\subsubsection{Actual Proof}
\[(P \rightarrow (\lnot R \rightarrow \lnot S)) \vee ((S \rightarrow (P \vee \lnot T)) \vee (\lnot Q \rightarrow R))\]
\[
\KMproof{
  \cbblk{
  \proofline{(P \rightarrow (\lnot R \rightarrow \lnot S)) \vee ((S \rightarrow (P \vee \lnot T)) \vee (\lnot Q \rightarrow R))}{}
  }{
    \proofline{\lnot((P \rightarrow (\lnot R \rightarrow \lnot S)) ...}{}
    \proofline{...\vee ((S \rightarrow (P \vee \lnot T)) \vee (\lnot Q \rightarrow R)))}{1,ID}
    \proofline{\lnot (P \rightarrow (\lnot R \rightarrow \lnot S))}{2,Theorem1}
    \proofline{\lnot ((S \rightarrow (P \vee \lnot T)) \vee (\lnot Q \rightarrow R))}{2,Theorem1}
    \proofline{\lnot (S \rightarrow (P \vee \lnot T))}{5,Theorem1}
    \proofline{\lnot(\lnot Q \rightarrow R)}{5,Theorem1}
    \proofline{\lnot( Q \vee R)}{7,Theorem2}
    \proofline{\lnot Q}{8,Theorem1}
    \proofline{\lnot R}{8,Theorem1}
    \proofline{\lnot(\lnot S \vee (P \vee \lnot T))}{6,Theorem2}
    \proofline{S}{11,Theorem1}
    \proofline{\lnot (P \vee \lnot T)}{11,Theorem1}
    \proofline{\lnot P}{13,Theorem1}
    \proofline{T}{13,Theorem1}
    \proofline{\lnot ( \lnot P \vee (\lnot R \rightarrow \lnot S))}{3,Theorem2}
    \proofline{P}{16,Theorem 1}
    \proofline{\lnot P}{14}
  }
}
\]
\section{Premises}
\subsection{Premises are Contradictions}

if our premises form a contradiction, then
  \[ \lnot ( P_{1} \wedge P_{2} \dots \wedge P_{N}) \rightarrow T \]
we turn it into a tautology, then we do what we are used to do:

we would then use a semantic tableau to show contradiction will always occurif we start with:

  \[ \lnot ( P_{1} \wedge P_{2} \dots \wedge P_{N}) : F \]

this will end up being:

  \[ ( P_{1} \wedge P_{2} \dots \wedge P_{N}) : T\]
  we would go through and split all the elements inside the ANDs
  showing that our starting logic equation is indeed a tautology.\\
  Therefore showing the contradiction.\\

\bigskip

we would also be able to use KM derivation
indeed we would start with
  \[ Show \lnot ( P_{1} \wedge P_{2} \dots \wedge P_{N}) \]
  \[( P_{1} \wedge P_{2} \dots \wedge P_{N}) 1,ID\] 
  
if the premises are contradictory, we would end up with a contradiction\\
proving our statement is true, therefore a tautology\\
\bigskip\\
Since we took the negation of our premises, we have then proven that our premises form a contradiction\\

\subsection{Premises are Tautologies}
Having a premise that is a tautology will not cause a problem for proving things, but the logical conclusion must also be a tautology by itself.\\
I have to say that this makes the premises rather useless in the proof.\\
\subsection{Some Practice}
Consider the following set of premises: ``Sales of houses fall off if interest rates rise. Auctioneers are not happy if sales of houses fall off. Interest rates are rising. Auctioneers are happy.
\[
P: \text{Sales of houses fall off} 
\]
\[
Q: \text{Interest rates rise}
\]
\[
R: \text{Auctioneers are not happy}
\]
\begin{equation} 
Q\rightarrow P 
\end{equation}
\begin{equation} 
P \rightarrow R
\end{equation}
\begin{equation} 
Q
\end{equation}
\begin{equation} 
\lnot R
\end{equation}
\subsubsection{Truth Table}
if this set of premises form a contradiction then that means
  \[( (Q\rightarrow P) \wedge (P \rightarrow R) \wedge Q \wedge \lnot R )
  \]
  is always false \\
\begin {displaymath}
\begin {array} {|c c c|c|c|c|c|c|c|}
P & Q & R & (Q \rightarrow P) & 
(P \rightarrow R)&
Q&
\lnot R&
(Q\rightarrow P) \wedge (P \rightarrow R) \wedge Q \wedge \lnot R \\
\hline
F & F & F & T & T & F & T & F\\
F & F & T & T & T & F & F & F\\
F & T & F & F & T & T & T & F\\
F & T & T & F & T & T & F & F\\
T & F & F & T & F & F & T & F\\
T & F & T & T & T & F & F & F\\
T & T & F & T & F & T & T & F\\
T & T & T & T & T & T & F & F\\
\end{array}
\end{displaymath}
as the last column says, it is indeed all false \\
\pagebreak
\subsubsection{KM derivation}
  \[\lnot ( (Q\rightarrow P) \wedge (P \rightarrow R) \wedge Q \wedge \lnot R )
  \]
  is always true\\
\[
\KMproof{
  \cbblk{
  \proofline{\lnot ( (Q\rightarrow P) \wedge (P \rightarrow f) \wedge Q \wedge \lnot R )}{}
  }{
    \proofline{( (Q\rightarrow P) \wedge (P \rightarrow R) \wedge Q \wedge \lnot R )}{1,ID}
    \proofline{(Q \rightarrow P)}{2,SIMP}
    \proofline{(P \rightarrow R)}{2,SIMP}
    \proofline{Q} {2,SIMP}
    \proofline{\lnot R}{2,SIMP}
    \proofline{P}{3,5,MP}
    \proofline{\lnot P}{4,6,MT}
  }
}
\]
therefore 
  \[( (Q\rightarrow P) \wedge (P \rightarrow R) \wedge Q \wedge \lnot R )
  \]
  is a contradiction\\
  \pagebreak
\section{Normal Form}
  Well, knowing that Q1,Q5,Q6,Q7 and Q9 are tautologies, their CNF is just 1\\
  I will only do the CNF for the 4 other questions\\
\subsection{Q2}
\[(P \wedge Q) \leftrightarrow (P \rightarrow Q)\]
\[(P \wedge Q) \leftrightarrow (\lnot P \vee Q)\]
\[((P \wedge Q) \rightarrow (\lnot P \vee Q)) \wedge 
  ((\lnot P \vee Q) \rightarrow (P \wedge Q)) \]
\[(\lnot (P \wedge Q)\vee (\lnot P \vee Q)) \wedge 
  (\lnot (\lnot P \vee Q) \vee (P \wedge Q)) \]
\[(\lnot P \vee \lnot Q)\vee (\lnot P \vee Q)) \wedge 
  ((P \wedge \lnot Q) \vee (P \wedge Q)) \]
\[(\lnot P \vee \lnot Q\vee \lnot P \vee Q) \wedge 
  ((P \wedge \lnot Q) \vee (P \wedge Q)) \]
\[ (P \wedge \lnot Q) \vee (P \wedge Q) \]
\[ ((P \wedge \lnot Q) \vee P) \wedge ((P \wedge \lnot Q)  \vee Q) \]
\[ ((P \vee P) \wedge (\lnot Q \vee P) \wedge ((P \vee Q)\wedge 
  (\lnot Q \vee Q) \]
\[ P \wedge (\lnot Q \vee P) \wedge ((P \vee Q)\]
\[P\]
\subsection{Q3}
\[(\lnot P \vee Q) \rightarrow (P \rightarrow \lnot Q)\]
\[\lnot (\lnot P \vee Q) \vee (P \rightarrow \lnot Q)\]
\[\lnot (\lnot P \vee Q) \vee (\lnot P \vee \lnot Q)\]
\[(P \wedge \lnot Q) \vee (\lnot P \vee \lnot Q)\]
\[(P \vee (\lnot P \vee \lnot Q)) \wedge ((\lnot Q) \vee (\lnot P \vee \lnot Q))\]
\[ \lnot P \vee \lnot Q\]


\subsection{Q4}
\[(((P \rightarrow Q) \rightarrow P) \rightarrow Q)\]
\[(((\lnot P \vee Q) \rightarrow P) \rightarrow Q)\]
\[((\lnot (\lnot P \vee Q) \vee P) \rightarrow Q)\]
\[(\lnot (\lnot (\lnot P \vee Q) \vee P) \vee Q)\]
\[(\lnot ((P \wedge \lnot Q) \vee P) \vee Q)\]
\[(\lnot ((P \vee P) \wedge (\lnot Q \vee P)) \vee Q)\]
\[((\lnot (P \vee P) \vee \lnot (\lnot Q \vee P)) \vee Q)\]
\[((\lnot P \vee (Q \wedge \lnot P)) \vee Q)\]
\[(((\lnot P \vee Q )\wedge (\lnot P \vee \lnot P)) \vee Q)\]
\[(((\lnot P \vee Q )\wedge \lnot P) \vee Q)\]
\[(\lnot P \vee Q )\vee Q) \wedge (\lnot P \vee Q)\]
\[(\lnot P \vee Q) \wedge (\lnot P \vee Q)\]
\[\lnot P \vee Q\]



\subsection{Q8}

\[(((P \rightarrow Q) \rightarrow R) \rightarrow S) \leftrightarrow (P \rightarrow (Q \rightarrow (R \rightarrow S)))\]
\[(((\lnot P \vee Q) \rightarrow R) \rightarrow S) \leftrightarrow (P \rightarrow (Q \rightarrow (\lnot R \vee S)))\]
\[((\lnot (\lnot P \vee Q) \vee R) \rightarrow S) \leftrightarrow (P \rightarrow (\lnot Q \vee (\lnot R \vee S)))\]
\[(\lnot (\lnot (\lnot P \vee Q) \vee R) \vee S) \leftrightarrow (\lnot P \vee (\lnot Q \vee (\lnot R \vee S)))\]
\[(\lnot (\lnot (\lnot P \vee Q) \vee R) \vee S) \leftrightarrow (\lnot P \vee \lnot Q \vee \lnot R \vee S)\]
\[((\lnot (\lnot (\lnot P \vee Q) \vee R) \vee S) \rightarrow (\lnot P \vee \lnot Q \vee \lnot R \vee S))\wedge ((\lnot P \vee \lnot Q \vee \lnot R \vee S)\rightarrow (\lnot (\lnot (\lnot P \vee Q) \vee R) \vee S))\]
\[(\lnot (\lnot (\lnot (\lnot P \vee Q) \vee R) \vee S) \vee (\lnot P \vee \lnot Q \vee \lnot R \vee S))\wedge (\lnot (\lnot P \vee \lnot Q \vee \lnot R \vee S)\vee (\lnot (\lnot (\lnot P \vee Q) \vee R) \vee S))\]
\[(\lnot (\lnot ((P \wedge \lnot Q) \vee R) \vee S) \vee (\lnot P \vee \lnot Q \vee \lnot R \vee S))\wedge (\lnot (\lnot P \vee \lnot Q \vee \lnot R \vee S)\vee (\lnot ((P \wedge \lnot Q) \vee R) \vee S))\]
\[(\lnot ((\lnot(P \wedge \lnot Q) \wedge \lnot R) \vee S) \vee (\lnot P \vee \lnot Q \vee \lnot R \vee S))\wedge (\lnot (\lnot P \vee \lnot Q \vee \lnot R \vee S)\vee (\lnot ((P \wedge \lnot Q) \vee R) \vee S))\]
\bigskip\\
now only simplifying left hand side\\
\[(\lnot ((\lnot(P \wedge \lnot Q) \wedge \lnot R) \vee S) \vee (\lnot P \vee \lnot Q \vee \lnot R \vee S))\]
\[(\lnot (((\lnot P \vee Q) \wedge \lnot R) \vee S) \vee (\lnot P \vee \lnot Q \vee \lnot R \vee S))\]
\[((\lnot((\lnot P \vee Q) \wedge \lnot R) \wedge \lnot S) \vee (\lnot P \vee \lnot Q \vee \lnot R \vee S))\]
\[((((P \wedge \lnot Q) \vee R) \wedge \lnot S) \vee (\lnot P \vee \lnot Q \vee \lnot R \vee S))\]
\[((((P \vee R ) \wedge (\lnot Q \vee R) ) \wedge \lnot S) \vee (\lnot P \vee \lnot Q \vee \lnot R \vee S))\]
\[(P \vee R \vee \lnot P \vee \lnot Q \vee \lnot R \vee S) \wedge (\lnot Q \vee R \vee \lnot P \vee \lnot Q \vee \lnot R \vee S ) \wedge (\lnot S \vee \lnot P \vee \lnot Q \vee \lnot R \vee S)\]
\[1\]
\bigskip\\
now right hand side...\\
\[((P \wedge Q \wedge R \wedge \lnot S)\vee (\lnot ((P \wedge \lnot Q) \vee R) \vee S))\]
\[((P \wedge Q \wedge R \wedge \lnot S)\vee ((\lnot (P \wedge \lnot Q) \wedge \lnot R) \vee S))\]
\[((P \wedge Q \wedge R \wedge \lnot S)\vee (((\lnot P \vee Q) \wedge \lnot R) \vee S))\]
\[((P \wedge Q \wedge R \wedge \lnot S)\vee (((\lnot P \wedge \lnot R )\vee (Q \wedge \lnot R) \vee S))\]
\[(P \wedge Q \wedge R \wedge \lnot S)\vee (\lnot P \wedge \lnot R )\vee (Q \wedge \lnot R) \vee S\]
\[(P \wedge Q \wedge R \wedge \lnot S)\vee (\lnot P \wedge \lnot R )\vee (Q \vee S )\wedge (\lnot R \vee S)\]
this is a sum of products, I am going to use my ECE124 knowledge to convert this in to POS (CNF)\\
\begin{karnaugh-map}[4][4][1][$P Q$][$R S$]
  \manualterms{1,1,0,1,1,1,1,1,0,0,0,1,1,1,1,1}
  \implicant{8}{9}
  \implicantedge{2}{2}{10}{10}
\end{karnaugh-map}\\
CNF form:\\
\[(P \vee \lnot R \vee S) \wedge (\lnot P \vee Q \vee S)\]\\
\pagebreak

\section{Generalized DeMorgan's Laws}
\subsection{Theorem 1}
 need to prove a theorem really quicky before I prove this:\\
\[\frac{\lnot (P_{1} \vee P_{2})}{\lnot P_{1}}\]
proof by models through semantic tableau:\\
\begin{forest}
[$\lnot (P_{1} \vee P_{2}):T \CommaPunct \lnot P_{1} : F$
  [$P_{1} : T$
    [$(P_{1} \vee P_{2}):F$
      [$(P_{1}:F) \CommaPunct (P_{2} : F)$
        [X]
      ]
    ]
  ]
]
\end{forest}\\


\setcounter{equation}{0}
using induction...\\
\subsection{Negation of Conjunction }
probably should check the base case with one operator (i = 2)\\
\begin{equation}
(P_{1} \vee P_{2}) \leftrightarrow \lnot (\lnot P_{1} \wedge \lnot P_{2})
\end{equation}
\subsubsection{Base Case Forward}
\[
\KMproof{
  \cbblk{
  \proofline{ (P_{1} \vee P_{2}) \rightarrow \lnot (\lnot P_{1} \wedge \lnot P_{2})}{}
  }{
    \proofline{(P_{1} \vee P_{2})}{1,CD}
     \cbblk{
     \proofline{\lnot (\lnot P_{1} \wedge \lnot P_{2})}{2,SUBDERI}
   }{
      \proofline{(\lnot P_{1} \wedge \lnot P_{2})}{3,ID}
      \proofline{\lnot P_{1}}{4,SIMP}
      \proofline{\lnot P_{2}}{4,SIMP}
      \proofline{P_{1}}{6,2,MTP}
    }
 } 
}
\]
\subsubsection{Base Case : Backward} 
\[
\KMproof{
  \cbblk{
  \proofline{ (P_{1} \vee P_{2}) \leftarrow \lnot (\lnot P_{1} \wedge \lnot P_{2})}{}
  }{
    \proofline{\lnot (\lnot P_{1} \wedge \lnot P_{2})}{1,CD}
     \cbblk{
     \proofline{(P_{1} \vee P_{2})}{2,subDer}
   }{
      \proofline{\lnot (P_{1} \vee P_{2})}{3,ID}
      \proofline{\lnot P_{1}}{4,THEOREM1}
      \proofline{\lnot P_{2}}{4,THEOREM1}
      \proofline{\lnot P_{1} \wedge \lnot P_{2}} {5,6,ADJ}
      \proofline{\lnot (\lnot P_{1} \wedge \lnot P_{2})}{2}
    } 
  }
}
\]

\subsubsection{N+1}
knowing that:\\
  \[\bigvee_{i = 1}^N P_i \leftrightarrow \lnot \bigwedge_{i = 1}^N (\lnot P_i)\]
show:\\
\[\bigvee_{i = 1}^{N+1} P_i \leftrightarrow \lnot \bigwedge_{i = 1}^{N+1} (\lnot P_i)\]
\textbf{Starting with left-hand side:}\\
\[\bigvee_{i = 1}^{N+1} P_i \]
\[(\bigvee_{i = 1}^{N} (P_i)) \vee P_{N+1}\]
applying the base case:\\
\[\lnot (\lnot \bigvee_{i = 1}^{N} (P_i) \wedge \lnot P_{N+1})\]
applying the assumption:\\
\[\lnot (\lnot \lnot \bigwedge_{i = 1}^{N} (\lnot P_i) \wedge \lnot P_{N+1})\]
\[\lnot (\bigwedge_{i = 1}^{N} (\lnot P_i) \wedge \lnot P_{N+1})\]
\[\lnot \bigwedge_{i = 1}^{N+1}(\lnot P_i)\]
\textbf{Starting with RHS:}\\
\[\lnot (\bigwedge_{i = 1}^{N+1} (\lnot P_i))\]
\[\lnot (\bigwedge_{i = 1}^{N} (\lnot P_i) \wedge (\lnot P_{N+1} )) \]
\[\lnot (\lnot \lnot \bigwedge_{i = 1}^{N} (\lnot P_i) \wedge (\lnot P_{N+1} )) \]
applying the base case here:\\
\[(\lnot \bigwedge_{i = 1}^{N} (\lnot P_i) \vee P_{N+1}) \]
applying the assumption:\\
\[(\bigvee_{i = 1}^{N} (P_i) \vee P_{N+1}) \]

\[\bigvee_{i = 1}^{N+1} (P_i) \blacksquare \]
\subsection{Negation of Disjunction}
\begin{equation}
(P_{1} \wedge  P_{2}) \leftrightarrow \lnot (\lnot P_{1} \vee \lnot P_{2})
\end{equation}
\subsubsection{Base Case Forward}

\[
\KMproof{
  \cbblk{
  \proofline {(P_{1} \wedge  P_{2}) \rightarrow \lnot (\lnot P_{1} \vee \lnot P_{2})}{}
  }{
     \proofline{(P_{1} \wedge P_{2})}{1,CD}
     \proofline{P_{1} }{3,SIMP}
     \proofline{P_{2} }{3,SIMP}
     \cbblk{
     \proofline{\lnot (\lnot P_{1} \vee \lnot P_{2})}{SUBDERI}
   }{
      \proofline{(\lnot P_{1} \vee \lnot P_{2})}{3,ID}
      \proofline{\lnot (\lnot  P_{1})}{3,DN}  
      \proofline{\lnot P_{2}}{6,7,MTP}
    }
  } 
}
\]
\subsubsection{Base Case Backward} 
\[
\KMproof{
  \cbblk{
  \proofline {(P_{1} \wedge  P_{2}) \leftarrow \lnot (\lnot P_{1} \vee \lnot P_{2})}{}
  }{
    \proofline{\lnot (\lnot P_{1} \vee \lnot P_{2})}{1,CD}
    \proofline{\lnot \lnot P_{1}}{2,THEOREM1}
    \proofline{\lnot \lnot P_{2}}{2,THEOREM1}
    \proofline{P_{1}}{6,DN}
    \proofline{P_{2}}{7,DN}
    \proofline{P_{1} \wedge P_{2}} {6,7,ADJ}
  }
}
\]
\subsubsection{N+1}
knowing that:\\
  \[\bigwedge_{i = 1}^N P_i \leftrightarrow \lnot \bigvee_{i = 1}^N (\lnot P_i)\]
show:\\
\[\bigwedge_{i = 1}^{N+1} P_i \leftrightarrow \lnot \bigvee_{i = 1}^{N+1} (\lnot P_i)\]

\textbf{Starting with left-hand side:}\\
\[\bigwedge_{i = 1}^{N+1} P_i \]
\[(\bigwedge_{i = 1}^{N} (P_i)) \wedge P_{N+1}\]
applying the base case:\\
\[\lnot (\lnot \bigwedge_{i = 1}^{N} (P_i) \vee \lnot P_{N+1})\]
applying the assumption:\\
\[\lnot (\lnot \lnot \bigvee_{i = 1}^{N} (\lnot P_i) \vee \lnot P_{N+1})\]
\[\lnot (\bigvee_{i = 1}^{N} (\lnot P_i) \vee \lnot P_{N+1})\]
\[\lnot \bigvee_{i = 1}^{N+1}(\lnot P_i) \blacksquare \]\\
\textbf{Starting with RHS:}\\
\[\lnot (\bigvee_{i = 1}^{N+1} (\lnot P_i))\]
\[\lnot (\bigvee_{i = 1}^{N} (\lnot P_i) \vee (\lnot P_{N+1} )) \]
\[\lnot (\lnot \lnot \bigvee_{i = 1}^{N} (\lnot P_i) \vee (\lnot P_{N+1} )) \]
applying the base case here:\\
\[(\lnot \bigvee_{i = 1}^{N} (\lnot P_i) \wedge P_{N+1}) \]
applying the assumption:\\
\[(\bigwedge_{i = 1}^{N} (P_i) \wedge P_{N+1}) \]

\[\bigwedge_{i = 1}^{N+1} (P_i) \blacksquare \]
\end{document}
%done?
